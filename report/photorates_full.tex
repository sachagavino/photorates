\documentclass[a4paper]{article}

\usepackage[english]{babel}
\usepackage[utf8]{inputenc}
\usepackage{amsmath}
\usepackage{caption}
\usepackage{subcaption}
\usepackage{graphicx}
\usepackage{float}
\usepackage{url}
\usepackage[colorinlistoftodos]{todonotes}
\usepackage[top=1in, bottom=1.25in, left=1.25in, right=1.25in]{geometry}

\title{Photorates by UV fields in protoplanetary disks} 
\date{\vspace{-10ex}}
\begin{document}
\maketitle

%%%%%SECTION%%%%%
%%%%%%%%%%%%%%%
\section{Definition}
The evolution of abundance of a species $X$ due to dissociation generated by UV photons in the gas phase is given by a first order differential equation of the form:

\begin{equation}
\label{eq:diff}%diff
	\frac{dn(X)}{dt} = K_d(Y)n(Y) - K_d(X)n(X)
\end{equation}

\noindent where $n(X)$ [$\mathrm{cm^{-3}}$] is the density of species $X$, $K_d(X)$ [$\mathrm{s^{-1}}$] is the photodissociation rate of species $X$, $n(Y)$ and $K_d(Y)$ are the density and the photodissociation rate of another species $Y$, respectively. 

\noindent Let's define the photodissociation rate $K_d$. The photon spectral density is

\begin{equation}
\label{eq:uv}%uv
	u_\nu = \frac{1}{c}I(\nu)
\end{equation}

\noindent where $I(\nu)$ is the photon-based intensity [$\mathrm{cm^{-3}s^{-1}}$] summed over all angles. The total number of UV photons absorbed per $\mathrm{cm^{3}}$ per second is then

\begin{equation}
\label{eq:total}%total 
	\int_\nu c k_d(\nu) u_\nu d\nu = \int_\nu k_d(\nu)I(\nu)d\nu
\end{equation}

\noindent Where the integral is done along the whole line. We define the photodissociation coefficient:

\begin{equation}
\label{eq:coef}%coef 
	k_d(\nu) = n_{ph}(\nu)\sigma_{diss}(X, \nu)
\end{equation}

\noindent so that Eq. \ref{eq:total} can be written as:

\begin{equation}
\label{eq:total2}%total2 
	\int_\nu  n_{ph}(\nu) \sigma_{diss}(X, \nu) c u_\nu d\nu = \int_\nu n_{ph}(\nu) \sigma_{diss}(X, \nu) I(\nu) d\nu
\end{equation}

\noindent where $n_{ph}(\nu)$ is the number of photons absorbed and $\sigma_{diss}(X, \nu)$ is the photodissocation cross-section. 

\noindent Therefore, we can define the probability $K_d(X)$ [$\mathrm{s^{-1}}$] that a single photon leads to a photodissociation event of a molecule $X$:


\begin{equation}
\label{eq:Kdef}%Kdef
	K_d(X) =  \int_\nu  \sigma_{diss}(X, \nu) I(\nu) d\nu
\end{equation}



\section{Source of radiation fields} %%%%%SECTION: SOURCE OF RADIATION FIELDS%%%%%
\noindent We want to define the value $K_d(X)$ of each species involved in Nautilus chemical network at specific time-steps and at every coordinates $(r,z)$ of the protoplanetary disk we want to study.  
The first tricky step is the modelling of the UV field. As we study disks with a flaring geometry, we assume that the disk surface is exposed to the star radiation. For this reason we consider two main sources of UV flux. One from the interstellar radiation field (ISRF), that will be computed vertically, the other from the central forming star computed radially. Fig.\ref{fig:field} shows the UV fields received by a molecule at coordinates $(r,z)$ in a disk structure as defined in the models. The photorates will be a result of both the stellar field and the interstellar radiation field.

\begin{figure}[H]
\centering
\includegraphics[width=1\textwidth]{images/field.png}
\caption{\label{fig:field} Elevation above the midplane as a function of the radius. The dotted black lines are scale heights of the gas (1$H$, 2$H$, 3$H$ and 4$H$). The red lines are radiation emitted from the central star and from the interstellar radiation field (ISRF). The coordinates $(r,z)$ represent a point at where both fields cross. We want to characterize the field in each point of the disk. }
\end{figure}

\noindent We use Draine field to characterize the ISRF, and a black body-like source for the star. This is likely to change soon since we want to model a more realistic stellar UV field, especially one that considers the important Lyman-alpha line. 

\section{UV penetration in the disk} %%%%%SECTION: UV PENETRATION IN THE DISK%%%%%
The second tricky step is to describe the penetration into the disk and the attenuation of the radiation field by the dust and gas. This is of great importance since characterizing the field in every parts of the disk is necessary to compute the photodissociation rates. As mentioned in the previous section, the field is a contribution of stellar and interstellar radiations:

\begin{equation}
\label{eq:sum}%sum
	I_{tot} =  I_{ISRF} + I_{*}
\end{equation}

\noindent Therefore, we must know the value of $I_{ISRF}$ and $I_{*}$ after being attenuated by the disk medium until it reaches the coordinates $(r,z)$. Resolving the radiative transfer equation in a non-scattering medium without emission, we write:

\begin{equation}
\label{eq:beer}%beer
	I =  I_0e^{-\tau}
\end{equation}

\noindent which is the Beer-Lambert law, where $I_0$ is the field before attenuation and $\tau$ is the optical depth due to the medium through which the light travelled. Given Eq. \ref{eq:sum} and Eq. \ref{eq:beer} we can define the local field at coordinates $(r,z)$ without diffusion as follows:


\begin{equation}
\label{eq:local}%local
	I_L(\nu, r, z) =   I_{ISRF}e^{-\tau^V} + I_{*}e^{-\tau^R}
\end{equation}

\noindent where $\tau^V$ is the optical depth due to the vertical attenuation of the ISRF by the medium above the local point and $\tau^R$ is the optical depth due to the radial attenuation of the stellar field by the medium through which the light travel from the star to the local point. As the optical depth is a contribution of both dust and gas, we write:


\begin{align}
\label{eqn:tau1}
\begin{split}
 \tau^V =   \tau^V_m + \tau^V_{d}
\\
 \tau^R=   \tau^R_m + \tau^R_{d}
\end{split}
\end{align}

\noindent where $m$ stands for molecules and $d$ for dust. So the local expression becomes:

\begin{equation}
\label{eq:local2}%local2
	I_L(\nu, r, z) =   I_{ISRF}e^{-(\tau^V_d + \tau^V_m)} + I_{*}e^{-(\tau^R_d + \tau^R_m)}
\end{equation}

\noindent thus, the photodissociation rate of a molecule should be written

 \begin{equation}
\label{eq:Kdef2}%Kdef2
	K_d(X) =  \int_\nu  \sigma_{diss}(X, \nu) I_L(\nu, r, z) d\nu
\end{equation}

\noindent Considering that we know both $I_{ISRF}$ and $I_{*}$, what we need to characterize are the optical depths from all contributions. While defining the vertical optical depth is trivial, the radial one is more tricky. We chose to write the ratio between the radial and vertical optical depths as follows:

 \begin{equation}
\label{eq:RV}%RV
	RV =  \frac{\tau^R}{\tau^V}
\end{equation}

\noindent We assume that in most cases, $RV >> 1$, in a few surface layers $RV \approx 1$ and that the case $RV << 1$ never happens. 
\noindent $\tau^V$ is trivial to characterize but $\tau^R$ is another story. In order to simplify the computation of $I_L(\nu, r, z)$ we write Eq. \ref{eq:local2} as follows:

\begin{equation}
\label{eq:local3}%local3
	I_L(\nu, r, z) =   I_{ISRF}e^{-(\tau^V_d + \tau^V_m)} + I_{*}e^{-RV(\tau^V_d + \tau^V_m)}
\end{equation}

\noindent We need to define $RV$. We already have computed:


\begin{equation}
\label{eq:local_d}%local_d
	I_d(\nu, r, z) =   I_{ISRF}e^{-\tau^V_d} + I_{*}e^{-RV\tau^V_d}
\end{equation}

\noindent so considering we also computed $\tau^V_d$ and that we know both $I_{*}$ and $I_{ISRF}$, we can get the ratio:


\begin{equation}
\label{eq:ratio}%ratio
	RV =   -\frac{1}{\tau_d^V}ln\bigg(\frac{I_d(\nu, r, z) - I_{ISRF}e^{-\tau^V_d}}{I_*}\bigg)
\end{equation}

\noindent Assuming $RV$ is the same for the molecular and dust opacities we can define the local flux $I_L(\nu, r, z)$ in each cells of the disk. 

\section{Optical depths} %%%%%SECTION: OPTICAL DEPTHS%%%%%
The general definition of the opacity is

\begin{equation}
\label{eq:tau}%tau
	\tau(\nu) =  \int_l k(l, \nu)  dl
\end{equation}

\noindent where $k(l, \nu)$ is an absorption coefficient and the integration is made along a path of light. 
\noindent Both dust and gas contribute to the UV attenuation and we give the definition.

\subsection{the molecular contribution} %%%%%SUBSECTION: molecule depth%%%%%
For a line, the molecules contribute as follows:

\begin{equation}
\label{eq:tau_m}%tau_m
	\tau_m(\nu, l, z) =  \sum_X \int_l n(X, r, z) \sigma_{abs}(X, \nu) dl
\end{equation}

\noindent where $X \in \left\{ H2, CO, ... \right\}$ are the molecules that contribute significantly to the attenuation, $n(X, r,z)$ is the number density [$\mathrm{cm^{-3}}$] of species $X$ at coordinates $(r,z)$ and $\sigma_{abs}(X, \nu)$ is the absorption cross-section of species $X$ at  wavelength $\nu$. Whether the radiation is from the interstellar medium or from the central star, the line $l$ over which we integrate is either the altitude $z$ above the considered point of coordinates $(r,z)$ or the radial distance $r$ between the central star and the considered point, respectively. 

\subsection{the dust contribution}%%%%%SUBSECTION: dust depth%%%%%
By definition, we use the same path over which we integrate and the absorption coefficient is different. We write: 

\begin{equation}
\label{eq:tau_d}%tau_d
	\tau_d(\nu, l, z) =  \sum_{i} \int_l \kappa_e(a_i, \nu) n_d(a_i, r,z) dl
\end{equation}

\noindent $\kappa_e(a_i, \nu)$ is the dust extinction coefficient and $n_d(a_i, r,z)$ is the number density [$\mathrm{cm^{-3}}$] of the grains of size $a_i$ at coordinates $(r,z)$. We use the extinction coefficient because when a beam light is intercepted by a dust grain, the radiation undergoes absorbtion and scattering processes. 

\noindent To define the extinction coefficient, we introduce the extinction efficiency:

\begin{equation}
\label{eq:Qe}%Qe
	Q_e(a_i, \nu) = \frac{\sigma_{e}(a_i, \nu)}{\sigma_{g}(a_i)}
\end{equation}

\noindent $\sigma_{e}(a_i, \nu)$ is the extinction cross-section for grains of size $a_i$ and $\sigma_{g}(a_i)$ is the geometrical cross-section. Since the extinction is a combined effect of absorption and scattering of light: 

\begin{equation}
\label{eq:sigma_e}%sigma_e
	\sigma_{e}(a_i, \nu) = \sigma_{abs}(a_i, \nu) + \sigma_{sca}(a_i, \nu)
\end{equation}

\noindent Consequently:

\begin{equation}
\label{eq:q_def}%q_def
	Q_{e}(a_i, \nu) = Q_{abs}(a_i, \nu) + Q_{sca}(a_i, \nu)
\end{equation}

\noindent and the extinction coefficient is defined as:

\begin{equation}
\label{eq:k}%k
	\kappa_e(a_i, \nu) = \sigma_{e}(a_i, \nu) = \sigma_{g}(a_i) Q_e(a_i, \nu)
\end{equation}

\noindent Given the fact that the geometrical cross-section is the squared grain size, Eq. \ref{eq:tau_d} is finally:

 \begin{equation}
\label{eq:tau_d2}%tau_d2
	\tau_d(\nu, r, z) =  \sum_{i} \int_l  n_d(a_i, r,z) a_i^2 Q_e(a_i, \nu)dl
\end{equation}

\section{Computation in Nautilus}%%%%%SECTION: COMPUTATION IN NAUTILUS%%%%%
We add a subroutine in Nautilus code that extracts outputs at macroscopic time-steps in order to recalculate the opacity at each coordinates and update the photorates $K_d(X)$ for molecules of the network.
The cross sections values expressed in $\mathrm{cm^{2}}$ can be found in the Leiden University website\footnote{\url{https://home.strw.leidenuniv.nl/~ewine/photo/index.php?file=cross_sections.php}}. They are collected from experimental and theoretical studies and are given for a large range of wavelengths. We use them for every species of interest in Nautilus.

\noindent The problem is very simplified at the surface of the disk. We simply consider a sharp frontier between $4H$ and the ISM. We consider that the medium above $4H$ do not contribute to any attenuation. 













\newpage 
\begin{thebibliography}{9}

\bibitem{heays} 
A. N. Heays et al. 
\textit{Photodissociation and photoionisation of atoms and molecules
of astrophysical interest}. 
A\&A, 2017.
 
\bibitem{springer} 
Walter J. Maciel. 
\textit{Astrophysics of the interstellar medium}. 
Springer, Chap.6 , 2013.

\bibitem{chapillon} 

\textit{new-uv2}. 


\end{thebibliography}

\end{document}